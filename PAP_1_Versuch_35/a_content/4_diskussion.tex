Verschiedene Frequenzen können unterschiedlich gut Elektronen aus der Kathode
auslösen. Anhand dieses Zusammenhangs lässt sich die Plackkonstante bestimmen.
In unserem Versuch haben wir für die Frequenzen des Hg-Spektrums den Fotostrom
gemessen. Daraus enstanden 5 Messreihen aus welchen die Sperrspannungen $U_s$
ausgelesen wurden. Daraus konnte der Zusammenhang zur Frequenz hergestellt werden
und die Plackkonstante bestimmt werden. Dieser Zusammenhang ist genau $\tfrac{h}{e}$.\\
\smallbreak
Die bestimmte Planckkonstante ist $ h= (6,52 \pm 0,21 )\cdot 10^{-34} \text{Js}$ was im Vergleich
zu dem Literaturwert von $h_{Lit} =6,626 070 15 \cdot 10^{-34} \text{Js}$ einer Abweichung von
$0,5 \sigma$ eentspricht. Unsere Messungkann  als genau angenommen werden,
da sie sich im $1 \sigma$ Bereich befindet und der ausreichen klein ist (3,2 \%).
Dieses Egebnis ist dennoch kritisch zu betrachten, da durch das händische Einzeichnen der
Trendgeraden große Fehler entstehen können. Ebenfalls spielen Fehlerquellen
wie die Sauberkeit derr Prismen, die Genauigkeit des eingestellten Intensitätsmaximum oder
die Genauigkeit der Gegenspannung ein Rolle. Diese wurden mit Blick auf die Abweichung
ausreichen berücksichtigt. \\
\smallbreak
Die Versuchdurchführung lässt Raum für Verbesserungen. Die größte Fehlerquelle ist
das händische Einzeichnen. Dort würde es das Ergebnis verbessen, wenn man kleinschrittigere Messungen
durchführt oder die Trendgeraden mittels eines Computerprogramms ermitteln lässt. \\
Ebenfalls könnten Lampen verwendet werden, welche mehr isolierte Spektrallinien produzieren
wodurch mehr Messreihen für Frequenzen entstehen.
