\section{Aufgabe I}

Zuerst werden die Tabellen aus dem Messprotokoll erngänzt. Dafür wird
$U_{I0} = U_I - U_0$ wobei $U_0$ der jeweilige Unterstrom ist.  Da ungefähre gilt
$I \propto U^2$ wird hier noch $\sqrt{U_{I0}}$ berechnet.

Für die Fehler gilt, $\Delta U_{Mult}$ als Fehler des Multimeters:

\begin{equation}
    \Delta U_{I0} = \sqrt{(\Delta U_{Mult})^2 + (\Delta U_I)^2} 
\end{equation}

\begin{equation}
    \Delta \left(\sqrt{U_I  - U_{I0}}\right) = \sqrt{\left(\frac{1}{2\sqrt{U_I - U_{I0}}} \Delta U_I\right)^2 + \left(-\frac{1}{2\sqrt{U_I - U_{I0}}} \Delta U_{I0}\right)^2}
\end{equation}
\begin{table}[ht]
    \centering
    \caption{Gemessene Spannungen UV}
    \begin{tabular}{r | r | r | r}
    \toprule
    $U$[mV] & $U_I$[mV] & $U_{I0}$[mV] & $\sqrt{U_I  - U_{I0}}[\sqrt{\text{mV}}]$\\
    \midrule
    $0$     &   $6030 \pm 25$   & $6072 \pm 25$& $77,92 \pm 0,16$ \\
    $-100$  &   $5510 \pm 24$   & $5552 \pm 24$& $74,5 \pm 0,16$\\
    $-200$  &   $5040 \pm 23$   & $5082 \pm 23$& $71,29 \pm 0,16$\\
    $-300$  &   $4530 \pm 21$   & $4572 \pm 21$& $67,61 \pm 0,16$\\
    $-400$  &   $4070 \pm 20$   & $4112 \pm 20$& $64,12\pm 0,16$ \\
    $-500$  &   $3650 \pm 16$   & $3692 \pm 16$& $60,76 \pm 0,13$ \\
    $-600$  &   $3204 \pm 14$   & $3246 \pm 14$& $56,97 \pm 0,12$ \\
    $-700$  &   $2816 \pm 12$   & $2858 \pm 12$& $53,46 \pm 0,11$ \\
    $-800$  &   $2447 \pm 11$   & $2489 \pm 11$& $49,89 \pm 0,11$\\
    $-900$  &   $2087  \pm 9$   & $2129 \pm 9$& $46,14 \pm 0,10$\\
    $-1000$ &   $1737 \pm 8$    & $1779 \pm 8$& $42,17 \pm 0,09$\\
    $-1100$ &   $1438 \pm 7$    & $1480\pm 7$& $38,47 \pm 0,09$\\
    $-1200$ &   $1130 \pm 6$    & $1172 \pm 6$& $34,23 \pm 0,08$\\
    $-1300$ &   $887 \pm 5$     & $929 \pm 5$&$ 30,5 \pm 0,08$\\
    $-1400$ &   $664 \pm4$      & $706 \pm 4$& $26,56 \pm 0,07$\\
    $-1500$ &   $461 \pm 3$     & $503 \pm 3$& $22,42 \pm 0,06$\\
    $-1600$ &   $307,2 \pm 1,3$ & $348,8 \pm 1,4$& $18,68 \pm 0,04$\\
    $-1700$ &   $181,5 \pm 1,0$ & $223,1 \pm 1,1$& $14,94 \pm 0,04$\\
    $-1800$ &   $83,4 \pm 0,7$  & $125,0 \pm 1,0$& $ 11,18 \pm 0,04$\\
    $-1900$ &   $18,8 \pm 0,5$  & $60,4 \pm 0,8$& $7,77 \pm 0,05$\\
    \bottomrule
    \multicolumn{4}{c}{$U_{I0} = -41,6$ , $\Delta U_{I0} = 0,6$}
    
    \end{tabular}
    
\end{table}
\newpage

% Table generated by Excel2LaTeX from sheet 'Sheet1'
\begin{table}[ht]
  \centering
  \caption{Gemessene Spannungen Violett}
    \begin{tabular}{r | r | r | r}
    \toprule
    $U$[mV] & $U_I$[mV] & $U_{I0}$[mV] & $\sqrt{U_I  - U_{I0}}[\sqrt{\text{mV}}]$\\
    \midrule
    $300 $  & $6650 \pm 27$ & $6684 \pm  27$ & $81.75 \pm  0.16$ \\
    $200$   &$ 6180 \pm 25$ &$ 6214 \pm  25$ & $78.83 \pm  0.16$ \\
    $100$   & $5660\pm  24$ & $5694 \pm  24$ & $75.46 \pm  0.16$ \\
    $0$     & $5140\pm  23$ & $5174 \pm  23$ & $71.93 \pm  0.16$ \\
    $-100$  & $4650 \pm  22$ & $4684 \pm 22$ &$ 68.44 \pm  0.16$\\
    $-200 $ & $4130\pm  20 $& $4164 \pm  20$ & $64.53 \pm  0.16$ \\
    $-300$  & $3650\pm  16$ & $3684 \pm  16$ & $60.69\pm  0.13 $\\
    $-400$  & $3228\pm  14$ & $3262 \pm  14$ &$ 57.11 \pm  0.12$ \\
   $ -500$  & $2816\pm  12$ & $2850 \pm  12$ & $53.38 \pm  0.12 $\\
    $-600$  & $2398\pm  11$ & $2432 \pm  11$ & $49.31 \pm  0.11$ \\
    $-700$  & $2021\pm  9$ & $2055 \pm  9$ & $45.33 \pm  0.10 $\\
    $-800$  & $1652\pm  8$ & $1686 \pm  8$ & $41.06 \pm  0.09$ \\
    $-900$  & $1318\pm  6$ & $1352 \pm  6$ & $36.77 \pm  0.09$ \\
    $-1000 $& $1014\pm  5$ & $1048 \pm  5$ & $32.37\pm  0.08$\\
    $-1100$ & $763\pm  4$ & $797 \pm  4$ & $28.23 \pm  0.07$ \\
   $ -1200$ & $520\pm  3$  & $554 \pm  3$ & $23.53 \pm  0.07$ \\
    $-1300$ & $335.8 \pm  1.3$ & $369.6 \pm  1.5$ & $19.22 \pm  0.04$ \\
    $-1400$ & $191.2 \pm  1.0$ & $225.0 \pm  1.1$& $15.00 \pm  0.04$ \\
    $-1500$ & $80.6  \pm  0.7$ & $114.4 \pm  1.0$ & $10.70 \pm  0.04$ \\
    $-1600$ & $13.5  \pm  0.5$ & $47.3  \pm  0.8 $&$ 6.88 \pm  0.06$\\
    \bottomrule
    \multicolumn{4}{c}{$U_{I0} = -33,8$ , $\Delta U_{I0} = 0,6$}
    \end{tabular}
    
\end{table}%

% Table generated by Excel2LaTeX from sheet 'Sheet1'
\begin{table}[h!]
  \centering
  \caption{Gemessene Spannungen Blau}
    \begin{tabular}{r | r | r | r}
    \toprule
    $U$[mV] & $U_I$[mV] & $U_{I0}$[mV] & $\sqrt{U_I  - U_{I0}}[\sqrt{\text{mV}}]$\\
    \midrule
    300   & $8500 \pm  31$ &$ 8554.4 \pm 31$ & $92.49 \pm 0.17$ \\
    200   & $7820 \pm 30$ & $7874.4 \pm 30$ &$ 88.74  \pm0.17$ \\
    100   & $7150 \pm28$ &$ 7204.4 \pm 28$ & $84.88 \pm0.16$\\
    0     & $6450 \pm 26$ & $6504.4 \pm 26$ & $80.65  \pm0.16$ \\
    -100  & $5760 \pm 24$ & $5814.4 \pm 24$ & $76.25 \pm 0.16$ \\
    -200  & $5160 \pm 23$ & $5214.4 \pm 23$ & $72.21 \pm 0.16$\\
    -300  & $4540 \pm 21$ & $4594.4 \pm 21$ & $67.78 \pm 0.16$ \\
    -400  & $3930 \pm 17$ & $3984.4 \pm 17$ & $63.12 \pm0.13$ \\
    -500  & $3355 \pm 14$ & $3409.4 \pm 14$ & $58.39 \pm0.12$ \\
    -600  & $2806 \pm 12$ & $2860.4 \pm 12$ & $53.48 \pm 0.11$ \\
    -700  & $2303 \pm 10$ & $2357.4 \pm 10$ &$ 48.55 \pm 0.11$ \\
    -800  &$ 1796 \pm 8$& $1850.4 \pm 8$ &$ 43.02  \pm0.09$ \\
    -900  & $1371\pm 6 $& $1425.4 \pm 7$ & $37.75  \pm0.09$ \\
    -1000 & $964 \pm 5$ & $1018.4 \pm 5 $& $31.91 \pm 0.08$ \\
    -1100 & $623 \pm 3$ & $677.4 \pm 4$ & $26.03 \pm0.07$\\
    -1200 & $350.7  \pm1.4$ & $405.1 \pm 1,5$ &$ 20.13 \pm 0.04$ \\
    -1300 & $147.5 \pm 0.9$ & $201.9 \pm 1.1$& $14.21 \pm0.04$ \\
    -1400 & $22.3  \pm 0.6$ & $76.7  \pm 0.8$ & $8.76 \pm 0.05$ \\
    \bottomrule
    \multicolumn{4}{c}{$U_{I0} = -54,4$ , $\Delta U_{I0} = 0,6$}
    \end{tabular}%
\end{table}%
\begin{table}[h!]
  \centering
  \caption{Gemessene Spannungen Grün}
    \begin{tabular}{r | r | r | r}
        \toprule
    $U$[mV] & $U_I$[mV] & $U_{I0}$[mV] & $\sqrt{U_I  - U_{I0}}[\sqrt{\text{mV}}]$\\
    \midrule
    $300$&$7170 \pm 28$ & $7198 \pm 28$ & $84.84 \pm 0.16$ \\
    $200$&$6390 \pm 26$ & $6418 \pm 26$ & $80.11 \pm 0.16$ \\
    $100$&$5600\pm 24 $   &$ 5628 \pm 24$ & $75.02 \pm 0.16$ \\
    $0$&$4790 \pm 22$ & $4818\pm 22$ & $69.41\pm 0.16 $\\
    $-100$&$4080 \pm 20 $ & $4108 \pm 20$ & $64.10 \pm 0.16 $\\
    $-200$&$3361  \pm 14$ & $3389 \pm 14$ & $58.22 \pm 0.12$ \\
    $-300$&$2678 \pm 12$ & $2706 \pm 12$ & $52.02 \pm 0.11 $\\
    $-400$&$1989 \pm 9 $& $2017 \pm 9$ & $44.91\pm 0.10$ \\
    $-500$&$1428 \pm 7$ & $1453 \pm 7$ & $38.16 \pm 0.09$ \\
    $-600$&$893 \pm 5$ & $921 \pm 5$ & $30.35 \pm 0.08$ \\
    $-700$&$444.0 \pm 2.8$ & $472.3 \pm 2.8$ & $21.73 \pm 0.07$ \\
    $-800$&$146.400 \pm 0.9$ & $174.7 \pm 1.0$ & $13.22 \pm 0.04$ \\
    $-900$&$19.700 \pm 0.5$ & $48.0    \pm 0.8 $& $6.93 \pm 0.06 $\\
    \bottomrule
    \multicolumn{4}{c}{$U_{I0} = -28,3$ , $\Delta U_{I0} = 0,6$}
    \end{tabular}%
\end{table}%

% Table generated by Excel2LaTeX from sheet 'Sheet1'

\begin{table}[ht]
  \centering
  \caption{Gemessene Spannungen Gelb}
   \begin{tabular}{r | r | r | r}
    \toprule
    $U$[mV] & $U_I$[mV] & $U_{I0}$[mV] & $\sqrt{U_I  - U_{I0}}[\sqrt{\text{mV}}]$\\
    \midrule
    300   & $5600 \pm 24$ & $5611  \pm24$ & $74.95 \pm 0.16$ \\
    200   &$ 4880 \pm 22 $&$ 4898\pm 22$ & $69.98 \pm 0.16$ \\
    100   & $4110 \pm 20$ & $4128 \pm 20$& $64.25 \pm 0.16$\\
    0     & $3477 \pm 15$ & $3495 \pm 15$ & $59.11 \pm 0.13$ \\
    -100  & $2771 \pm 12 $& $2789 \pm 12$ & $52.81 \pm 0.11$ \\
    -200  & $2145 \pm 10$ &$ 2163 \pm 10$ & $46.50 \pm0.10 $\\
    -300  & $1575 \pm 7$& $1593\pm 7 $& $39.91 \pm 0.09$\\
    -400  & $1025 \pm 5$ & $1043 \pm 5 $& $32.29 \pm 0.08$ \\
    -500  & $595 \pm 3$ & $613 \pm 3$ & $24.75 \pm 0.07 $\\
    -600  & $282.2 \pm 1.2$ & $299.7 \pm 1.3$ &$ 17.31 \pm 0.04$ \\
    -700  & $86.4 \pm 0.7$ & $103.9 \pm 0.9$ & $10.19 \pm 0.04$ \\
    -800  & $17.5  \pm 0.5$ & $35.0  \pm 0.8 $& $5.92 \pm 0.07$ \\
    \bottomrule
    \multicolumn{4}{c}{$U_{I0} = -17,5$ , $\Delta U_{I0} = 0,5$}
    \end{tabular}
\end{table}%
\clearpage
\newpage
\mbox{~}
\clearpage
\newpage


Aus den jeweils zugehörigen Diagrammen 1-5 ergibt sich durch das Ablesen der Nullstelle
der Trendgeraden die Sperrspannung $U_s$. Hier gilt für den Fehler mit  $U_{s-Felher}$ als
Nullstelle der Fehlergeraden. 

\begin{equation}
    \Delta U_s = U_s - U_{s-Felher}
\end{equation}

\begin{table}[h]
    \centering
    \caption{Spersspannungen nach Wellenlänge}
    \begin{tabular}{c | c }
        \toprule
        Frequenz[THz] & Sperrspannung $U_s$[V]\\
        \midrule
        518,7 & $0,83 \pm 0,03$ \\
        549,0 & $0,96 \pm 0,02$ \\
        687,0 & $1,54 \pm 0,03$ \\
        740,2 & $1,74 \pm 0,03$ \\
        821,3 & $ 2,06 \pm 0,03$ \\
        \bottomrule
    \end{tabular}
\end{table}



Daraus wurde Diagramm 6 erstellt und mithilfe des Betrags der Steigung der Trend und Fehlergeraden die Planck-Konstante $h$ bestimmt:

Dabei muss die Steigung der Geraden noch mit der Elementarladung $e = 1,602 \cdot 10^{-19} \text{C}$
multipliziert werden.

Für die Steigung der Trendgeraden gilt:
\begin{equation}
    a_{Trend} = \left|\frac{0,825-2,08}{308} \frac{\text{V}}{\text{THz}}\right| = 4,07 \cdot 10^{-15} \tfrac{\text{V}}{\text{Hz}}
\end{equation}
Für die Steigung der Fehler Gerade gilt:
\begin{equation}
    a_{Fehler} = \left|\frac{0,805-2,04}{294}  \frac{\text{V}}{\text{THz}}\right| = 4,20 \cdot 10^{-15} \tfrac{\text{V}}{\text{Hz}}
\end{equation}

Dabei wurde die negative 

\[ \Rightarrow a_{Trend} = (4,07\pm0,13) \cdot 10^{-15}\tfrac{\text{Js}}{\text{C}}\]

\[ \Rightarrow \underline{\underline{h = (6,52 \pm 0,21 )\cdot 10^{-34} \text{Js}}}\]

Daraus folgt mit einem Literaturwert von $h_{Lit} =6,626 070 15 \cdot 10^{-34} \text{Js}$ und Gleichung \ref{eq:sigma}, eine
Abweichung von $0,5 \sigma$.

