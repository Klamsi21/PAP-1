
\section{Zusamenfassung}

Es wurden mehrere Messreihen durchgeführt. Zuerst wurden die
Absorptionsmaxima der Kaliumpermanganatlösung mithilfe des Gitterspektrometers bestimmt.\\
Anschließend wurde die Intensität in Abhängigkeit der Schichtdicke gemessen. Dabei fiel auf,
dass der Messwert zu der 3 cm Küvette nicht in die Reihe gepasst hat und auch nach mehrfachem
Reproduzieren ähnliche Ergebnisse lieferte. Deswegen wurde auf ein Fehler des Messgeräts geschlossen
und dieser Wert in der Auswerung ignoriert.\\ 
Zuletzt wurde die Intensität in Abhängigkeit der Konzentration gemessen. Dafür wurde eine Küvette
mit VE Wasser gefüllt, die Intensität gemessen und die Konzentration an Kaliumpermanganat erhöht.
Daraus ließ sich die Beer Gerade kontruieren.

\section{Vergleich der Extinktionskoeffizienten}

Bei dem Vergleich der Extinktionskoeffizienten fällt die große Abweichung von
$20 \sigma$ auf. \\
Da kein Literaturwert vorhanden ist, lässt sich nicht überprüfen, welcher Wert
genauer ist. Da diese Abweichung sehr groß sit lässt diese nur auf systematische
Fehler schließen. Einerseits käme als Fehlerquelle das fehlerhafte Gitterspektrometer in Frage,
andererseits lässt auch das Alter der Kaliumpermnanganatlösungen zur Lambertgeraden
Fehler in der Genauigkeit zu. Weitere Fehlerquellen sind die Computersoftware oder der Faser, welcher das Licht zum Gitterspektrometer leitet. Genau zu ermitteln was diesen Fehler verursacht ist aufgrund 
der Anzahl der Fehlerquellen und der hohen Abweichung nicht möglich.
