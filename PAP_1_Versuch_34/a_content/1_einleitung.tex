\section{Motivation}

Spektralphotometrie ist ein essentielles Messverfahren zu Bestimmung
von Stoffzusammensetzungen und -konzentration. Die Zentrale Rolle bei
der Methode spielt das Labmert-Beersche GEsetz. Die Spektralphotometrie
bildet dabei die Schnittstelle zwischen Chemie und Physik, welshaöb der
Fakus dieses Versuchs auf dem erlernen der Methode liegt. Dementsprechend
ist der Umgang mit der Spektralphotometrie eine zentrale interdiziplinäre Kompetenz.


\section{Messverfahren}
Für das Experiment wird ein Spektrophotometer, welches aus einer Lampe, einer Linse, einem Küvettenhalter und einem
Gitterspektrometer besteht. In dem Gitterspektrometer wird das Licht auf einen CCD Sensor gestreut und damit die
Intensität des Lichts gemessen.

\subsection{Messung des Absorptionsspektrums}
ZU Beginn wird die 6 cm Küvette mit fester Konzentration Kaliumpermanganat verwendet um mit dem Gitterspektrometer
die Wellenlängen der Absoptionsmaxima zu bestimmen. Zuvor wird das Spektrometer mit einer Dunkelmessung und 
einer Messung ohne Küvette kalibriert.

\subsection{Absorption in Abhängigkeit der Küvettenlänge}
Für diese Messung wurde die Menge des Absorbierten Lichts bei einer festen Wellenlänge und varrierter
Küvettenlänge gemessen. Dies wurde für jede Küvette fünf mal wiederholt.

\subsection{Absorption in Abhängigkeit der Konzentration}

Für diese Messung wurde bei konstanter Wellenlänge und variabler Konzentration gemessen. Dabei wurden jeweils fünf Messungen
durchgefüht und danach die Kaliumpermanganatkonzentration erhöht.



\section{Grundlagen aus der Physik}

\subsection{Küvette}
Bei dem Durchlaufen des Lichts durch die Küvette wird das Licht leicht gebeugt, da die Küvette nicht Fehlerfrei ist.
Aus dem Messen der Durchmesser der Lichtkegel mit Küvellte $D_{mK}$ und ohne Küvette $D_{oK}$ lässt sich die Ablenkung berechnen und eliminieren:
\begin{equation}
    I_{korr} = I \cdot \frac{D_{mK}^2}{D_{oK}^2}
    \label{eq:Ikorr}
\end{equation}

\subsection{Stoffmengee und Konzentration}
Ein mol enpricht $6,022 \cdot 10^{23}$ Teilchen. Die Konzentration beschreibt die Anzahl von
Teilchen pro Volumeen. Dabei ist die Einheit $\tfrac{mol}{l}$.
Um bei einem hinzugefügten Volumen $V_j$ der Kaliumpermanganatlösung die Konzentration
zu bestimmen wird mithilfe des Ausgabsvolumen $V_0$, den einzel Volumina $V_j$ und der 
Konzentration $\tilde{c}$ des konzentrierten Kaliumpermangantas folgende Formal angewendet:
\begin{equation}
    c = \tilde{c} \frac{\sum_{i=1}^n V_i}{V_0 + \sum_{i=1}^n V_i}
    \label{eq:Konzentration}
\end{equation}

\subsection{Das Lambertsche Absorptionsgesetz}
Das Lambertsche Absorptionsgesetz vermittelt das fundamentale Verständniss der Lichtabsorption
in Lösungen. Nach diesem nimmt die Intensität exponentiell über den Verlauf durch das Medium ab.
Daraus folgt:

\begin{equation}
    I = I_0 \cdot e^{-kl}
\end{equation}
Dabei sind:
\begin{itemize}
    \item $I_0$ die Ausgangsintensität
    \item $k$ den Absoptionskoeffizienten
    \item $l$ die durchschrittene Länge in der Lösung
\end{itemize}


\subsection{Das Gesetz von Beer}
In einer verdünnten Lösung ist der Absorptionkoeffizient $k'$ proportional zur Konzentration $c$.
Diese unterscheiden sich in einem Faktor $\epsilon$ dem molaren Extinktionskoeffizienten. Diese
ist eine Stoffkonstante, welche die Absorption des Lichts einer bestimmten Wellenlänge angibt.

\begin{equation}
    k' = \epsilon \cdot c
    \label{eq:dekAbsorp}
\end{equation}

Aus der Verknüpfung dieser Beiden Gesetze folgt das Lambert-Beersche Gesetz:
\begin{equation}
    I = I_0 \cdot 10^{-\epsilon k l}
\end{equation}
Bei der umrechnung in die Zehner-Basis wird $k' = \log(e) k$ verwendet.

\section{Standartabweichung}
Allgemein lässt sich die Abweichung eines Messwertes $x$ zu einem Literaturwert $x_{Lit}$ darstellen durch die Sigmaabweichung:

\begin{equation}
    \frac{|x-x_{Lit}|}{\Delta x} = k \sigma \ \ mit \ k \in \mathbb{R}
    \label{eq:sigma}
\end{equation}