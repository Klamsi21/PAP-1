\section{Aufgabe I}

Aus der massendifferenz der Elektroden und der Laufzeit der Elektrolyse konnten die Faraday-Konstanten der Massenab bzw zunahme bestimmt werden.
\[\boxed{F_{zu} = \SI{9,7\pm 0,6}{\cdot 10^{4}\ \tfrac{\coulomb}{\mole}}}\]

\[\boxed{F_{ab} = \SI{9,3\pm 0,5}{\cdot 10^{4}\ \tfrac{\coulomb}{\mole}}}\]
Die Messung der Massenzunahme weist eine Abweichung zum Literaturwert von $0,09\ \sigma$ auf, die der Massenabnamhe $0,7\ \sigma$.
Das zeigt, dass die Messung der Massenzunahme deutlich genauer ist. Die Ergebnisse sind jedoch kritisch zu betrachten.
Auffällig ist, dass die Fehler beider Werte sehr groß sind, was besondes auf den Fehler der Stromstärke über einen so langen Zeitraum zurückzuführen ist.
Zu verbessern wäre hier die Verwendung eines genaueren Netzteils eventuell mit einer Automatischen regulierung der Stromstärke. \\
Anzumerken ist jedoch, dass beide Werte eine Abweichung von $0,5\ \sigma$ voneinander haben, was ebenfalls auf die hohen Fehler zurückzuführen ist.
Eigentlich zu erwarten wäre auch, dass der Wert durch Massenabnahme genauer ist, da die Kupferionen einfacher in Lösung gehen als sich abzusetzen.
Beobachtet man im gegenzug die Massendifferenzen, passt die Beobachtung dort auf diese Annahme. Das deutet auf einen systematischen Fehler hin, welcher
auch darin liegen könnte, dass die Zelle noch kurz ohne Strom in dem Elektrolyt lag, was zu einer kurzen Rückreaktion geführt haben könnte.
Insgesamt lässt sich jedoch sagen, dass die Messung erfolgreich war und die Messung der Massenzunahme genauer war.

\section{Aufgabe II}

Aus der jeweiligen Volumendifferenz des Wasserstoffs und Sauerstoffs konnten mithilfe der Raumtemperatur und des Luftdrucks die Faraday-Konstanten bestimmt werden:
\[\boxed{F_O = \SI{9,9\pm 0,5}{\cdot 10^{4}\ \tfrac{\coulomb}{\mole}}}\]
Die durch Sauerstoff bestimmte Faraday-Konstante weicht um $0,5\ \sigma$ vom Literaturwert ab. Auffällig ist, dass diese zu groß ist.
Dies ist damit zu erklären, dass der Sauerstoff teilweise mit der Säure des Elektrolyts reagiert, was für eine zu geringe Volumendifferenz sorgt.
Das senkt die bestimmte Stoffmenge und erhöht dadurch die berechnete Faraday-Konstante.\\
Die Faraday-Konstante bestimmt durch Wasserstoff
\[\boxed{F_H = \SI{9,6\pm 0,5}{\cdot 10^{4}\ \tfrac{\coulomb}{\mole}}}\]
weicht um $0,10\ \sigma$ vom Literaturwert ab. Daher ist diese Messung als äußerst erfolgreich zu werten.
Hier sind bei beden Werte erneut größere Fehler zu beobachten, welche wie oben durch die lange Messzeit zu erklären sind.
Da beide Werte $0,4\ \sigma$ voneinander Abweichen und der Wert durch Sauerstoff wie zu erwarten zu hoch ist, ist diese Messung im gesamten als erfolgreich zu betrachen.
Um die Fehler zu reduzieren wäre wie bei der Kupferzelle ein genaueres Netzteil hilfreich.

