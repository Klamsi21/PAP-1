\section{Motivation}

Die Elektrolyse wurde erstmals im Jahr 1800 entdeckt durch Alessandro Volta. Dieser stellte die erste brauchbare Batterie her.
In den darauffolgenden Jahren konnten mithilfe der Elektrolyse erstmals elementare unedle Metalle hergestellt werden. \\
Im Alltag präsent ist die Elektrolyse in Batterien, aber auch in Aluminium.
Um Aluminium elementar extrahieren zu können wird die Elektrolyse verwendet. Aluminium ist in der heutigen Zeit ein weitverbreitetder
Werkstoff, da es sehr leicht, aber auch stabil ist.

\section{Messverfahren}
Der Verscuh besteht aus drei Aufbauten:\\
\begin{itemize}
    \item Einer Elektrolysezelle aus zwei Kupferplatten und einem Kupfer-Sulfat Elektrolyt. Dabei wird Strom an die Kupferplatten gelegt. Durch den Strom
    scheidet sich an einer Platte elementares Kupfer ab, wärend bei der anderen Kupfer als Ion in Lösung geht. Aus der Massendifferenz der Platten vor und nach der Elektrolyse kann die Faraday-Kosntante bestimmt werden.
    \item Einer Elektrolysezelle mit zwei Platin Elektroden und Wasser. Durch den Strom entstehen elementarer Sauer- und Wasserstoff. Durch die Volumendifferenz zum Moment vor der Elektrolyse
    lässt sich hier ebenfalls die Faraday-Konstante bestimmen.
    \item Einer Brennstoffzelle, welche mithilfe von elementarem Wasserstoff und Luftsauerstoff, durch Rückreaktion der obigen Elektrolyse, Strom generieren kann.
\end{itemize}



\section{Grundlagen aus der Physik}

\subsection{Elektrolyse Kupfersulfat}

\begin{align}
    F &= \frac{Q}{zm}M_{Mol} \\
    \Leftrightarrow F &=  \frac{It}{zm}M_{Mol}
\end{align}

\begin{equation}
    V_{Mol} = \frac{p_0}{p}\frac{T}{T_0}V_{Mol}^0
\end{equation}