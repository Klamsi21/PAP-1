\section{Aufgabe I}
In dem Ersten Veruschsaufbau besteht das Elektrolyt aus in Wasser gelöstem Kupfersulfat. Dieses liegt dissoziiert vor.
Das Heist, dass sich die Inonenbindung trennen können und die Ionen frei vorliegen. Zwischen Inonenbindung und freien Ionen liegt ein Gleichgewicht vor.

\[CuSO_4 \rightleftharpoons Cu ^{2+} + SO_4^{2-}\]

Legt man nun eine Spannung an die Kupferelektroden so wandern Kupfer-Ionen aus der Lösung an die Kathode und nehmen zwei Elektronen auf. DAbei werden sie zu elementarem Kupfer reduziert,
welcher an der Elektrode ausfällt. Simultan lösen sich an der Anode Kupferatome aus, geben zwei Elektronen ab und gehen als zweifach geladene Ionen in Lösung.
Insgesamt bleibt also die Menge an gelöstem Kupfer gleich, da für jedes ausgelöste Kupfer, genau ein elementares Kupfer ausfällt.
Da die Spannung niedrig genug ist reagiert das Sulfat nicht und effektiv laufen folgende Reaktionen an Anode und Kathode ab:
\begin{align}
    Kathode&:Cu^{2+} + 2e^- \quad\rightarrow  \quad Cu\\
    Anode &: Cu \quad\rightarrow  \quad Cu^{2+} + 2e^-
\end{align}

\subsection{Massenzunahme}

Bei der Kathode fällt Kupfer aus was für eine Massenzunahme sorgt. Daher wurde die Platte vorher und Nachher gewogen. Dabei gilt für die Massendifferent $m$:
\[ m= \SI{0,3021 \pm 0,0014}{\gram}\]

Dabei gilt für den Fehler:
\begin{equation}
    \Delta m = \sqrt{(\Delta m_{vor})^2 + (\Delta m_{nach})^2}
\end{equation}

Damit kann nun $F$ mit Gleichung \ref{eq:FCu} bestimmt werden.

\[\boxed{F = \SI{9,7\pm 0,6}{\cdot 10^{4}\ \tfrac{\coulomb}{\mole}}}\]

Der Fehler wurde berechnet durch:
\begin{equation}
    \Delta F = \sqrt{\left(\frac{t}{zm} M \Delta I\right)^2 + \left(\frac{I}{zm} M \Delta t\right)^2 + \left(\frac{It }{z m^2} M\Delta m\right)^2}
\end{equation}


\subsection{Massenabnahme}
Analog wird die Faraday-Konstante aus der Massenabnahme der Anode bestimmt:
\[ m = \SI{0,3141 \pm 0,0014}{\gram}\]

\[\boxed{F = \SI{9,3\pm 0,5}{\cdot 10^{4}\ \tfrac{\coulomb}{\mole}}}\]



\section{Aufgabe II}

Für Aufgabe zwei wird der zweite Aufbau verwendet. Durch das dissoziieren der Säure entstehen $H_3O^+$ Kationen. Da das $SO_4^{2-}$ was als Säurerest bleibt,
bei der angelegten Spannung nicht reagiert laufen folgende Reaktionen ab:

\begin{align*}
Kathode&: 4\,\mathrm{H_3O^+} + 4\,e^- \quad\rightarrow \quad2\,\mathrm{H_2} + 4\,\mathrm{H_2O} \\
Anode&: 6\,\mathrm{H_2O} \quad \rightarrow \quad\mathrm{O_2} + 4\,\mathrm{H_3O^+} + 4\,e^-
\end{align*}




Hier ist zu berücksichtigen, dass für das Molvolumen der Sättigungsdruck der Schwefelsäure berücksichtigt werden muss.
In dem Behältnis hatte das Elektrolyt eine Temperatur von $22,5 \pm 0,5\ ^\circ \text{C}$. Das entspricht einem Sättigungsdruck des Wassers von $\SI{19,8 \pm 0,6}{\text{Torr}}$.
Daraus kann dann der Druck mit Gleichung \ref{eq:p} berechnet werden:

\[ p = \SI{738.9 \pm 0.6}{\text{Torr} }\]

\begin{equation}
    \Delta p = \sqrt{(\Delta p_L)^2 + (\Delta p_D^{H_2SO_4})^2}
\end{equation}

Damit wird anschließend das Molvolumen bei Raumtemperatur $T = 23,0 \pm 0,5\ ^\circ \text{C}$ nach Gleichung \ref{eq:VMol} bestimmt.

\[V_{Mol} = \SI{25,00 \pm 0.04}{\tfrac{\litre}{\mole}}\]

Für den Fehler gilt hier:
\begin{equation}
    \Delta V_{Mol} = \sqrt{\left(\frac{p_0}{p} \cdot \frac{\Delta T}{T_0} \cdot V_{Mol}^0\right)^2 + \left(\frac{p_0}{p^2} \cdot \frac{T}{T_0} \cdot V_{Mol}^0\cdot  \Delta p\right)^2}
\end{equation}

\subsection{Sauerstoff}

Die gemessene Volumendifferenz des Sauerstoffs beträgt:

\[m =  \SI{21,30 \pm 0,14}{\gram}\]

Daraus wird zunächst die Stoffmenge des entstandenen Sauerstoffs bestimmt nach Gleichung \ref{eq:n}:

\[n = \SI{8,52 \pm 0,06}{\cdot 10 ^{-4}\mole}\]

Dabei wird der Fehler berechnet durch:
\begin{equation}
    \Delta n = \sqrt{\left(\frac{1}{V_{Mol}} \cdot \Delta V\right)^2 + \left((\frac{V}{(V_{Mol})^2}) \cdot \Delta V_{Mol}\right)^2}
\end{equation}

Bei der oxidation von Sauerstoff, gibt dieser jeweis 2 Elektronen ab. Da für elementaren Sauerstoff 2 Sauerstoff oxidieren müssen, werden hier insgesamt 4 Elektronen übertragen.
Deshalb gilt hier $z=4$. Daraus ergibt sich für $F$ nach Gleichungen \ref{eq:m} und \ref{eq:FCu}:

\[\boxed{F = \SI{9,9\pm 0,5}{\cdot 10^{4}\ \tfrac{\coulomb}{\mole}}}\]

Mit Fehler:
\begin{equation}
    \Delta F = \sqrt{\left(\frac{I}{zn}  \cdot \Delta t\right)^2 + \left(\frac{t}{zn} \cdot \Delta I\right)^2 + \left(\frac{t I}{z n^2} \cdot \Delta n\right)^2}
\end{equation}


\subsection{Wasserstoff}

Zwei $H_3O^+$ welches durch dissoziieren der Säure entsteht, nehmen insgesamt 2 Elektronen auf, wodurch Wasser und elementarer Wasserstoff entsteht.
Daraus ergibt sich $z=2$. Mit der Massendifferenz $m$, kann hier analog zum Sauerstoff die Faraday-Konstante $F$ bestimmt werden.

\[m = \SI{43,90 \pm 0,14}{\gram}\]

\[n = \SI{1,756 \pm 0,006}{\cdot 10 ^{-3}\mole}\]

\[\boxed{F = \SI{9,6\pm 0,5}{\cdot 10^{4}\ \tfrac{\coulomb}{\mole}}}\]
