\section{Aufgabe I}
In dem Ersten Veruschsaufbau besteht das Elektrolyt aus in Wasser gelöstem Kupfersulfat. Dieses liegt dissoziiert vor.
Das Heist, dass sich die Inonenbindung trennen können und die Ionen frei vorliegen. Zwischen Inonenbindung und freien Ionen liegt ein Gleichgewicht vor.

\[CuSO_4 \rightleftharpoons Cu ^{2+} + SO_4^{2-}\]

Legt man nun eine Spannung an die Kupferelektroden so wandern Kupfer-Ionen aus der Lösung an die Kathode und nehmen zwei Elektronen auf. DAbei werden sie zu elementarem Kupfer reduziert,
welcher an der Elektrode ausfällt. Simultan lösen sich an der Anode Kupferatome aus, geben zwei Elektronen ab und gehen als zweifach geladene Ionen in Lösung.
Insgesamt bleibt also die Menge an gelöstem Kupfer gleich, da für jedes ausgelöste Kupfer, genau ein elementares Kupfer ausfällt.
Da die Spannung niedrig genug ist reagiert das Sulfat nicht und effektiv laufen folgende Reaktionen an Anode und Kathode ab:
\begin{align}
    Kathode&:Cu^{2+} + 2e^- \quad\rightarrow  \quad Cu\\
    Anode &: Cu \quad\rightarrow  \quad Cu^{2+} + 2e^-
\end{align}

\subsection{Massenzunahme}

Bei der Kathode fällt Kupfer aus was für eine Massenzunahme sorgt. Daher wurde die Platte vorher und Nachher gewogen. Dabei gilt für die Massendifferent $m$:
\[ m= \SI{0,3021 \pm 0,0014}{\gram}\]

Dabei gilt für den Fehler:
\begin{equation}
    \Delta m = \sqrt{(\Delta m_{vor})^2 + (\Delta m_{nach})^2}
\end{equation}

Damit kann nun $F$ mit Gleichung \ref{eq:FCu} bestimmt werden.

\[F = \SI{9,7\pm 0,6}{\cdot 10^{4}\ \tfrac{\coulomb}{\mole}}\]

Der Fehler wurde berechnet durch:
\begin{equation}
    \Delta F = \sqrt{\left(\frac{t}{zm} M \Delta I\right)^2 + \left(\frac{I}{zm} M \Delta t\right)^2 + \left(\frac{It }{z m^2} M\Delta m\right)^2}
\end{equation}


\subsection{Massenabnahme}
Analog wird die Faraday-Konstante aus der Massenabnahme der Anode bestimmt:
\[ m = \SI{0,3141 \pm 0,0014}{\gram}\]

\[F = \SI{9,3\pm 0,5}{\cdot 10^{4}\ \tfrac{\coulomb}{\mole}}\]



\section{Aufgabe II}

\[ p = \SI{738.9 \pm 0.6}{\milli \text{bar} }\]

\begin{equation}
    \Delta p = \sqrt{(\Delta p_L)^2 + (\Delta p_D^{H_2SO_4})^2}
\end{equation}

\[V_{Mol} = \SI{25,00 \pm 0.04}{\tfrac{\litre}{\mole}}\]

\begin{equation}
    \sqrt{\left(\frac{p_0}{p} \cdot \frac{\Delta T}{T_0} \cdot V_0\right)^2 + \frac{p_0}{p^2} \cdot \frac{T}{T_0} \cdot V_0 \cdot 0.75 \cdot \Delta p}
\end{equation}

\subsection{Sauerstoff}

\[n = \SI{8,52 \pm 0,06}{\cdot 10 ^{-4}\mole}\]

\[F = \SI{10,8\pm 0,6}{\cdot 10^{4}\ \tfrac{\coulomb}{\mole}}\]


\subsection{Wasserstoff}

\[n = \SI{1,756 \pm 0,006}{\cdot 10 ^{-3}\mole}\]

\[F = \SI{9,6\pm 0,5}{\cdot 10^{4}\ \tfrac{\coulomb}{\mole}}\]
