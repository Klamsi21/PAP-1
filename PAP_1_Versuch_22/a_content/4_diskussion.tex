In diesem Versuch konnten mithilfe von Stoppuhren die Fall und Steigzeiten der Öltröpfchen im Spannungsfeld bestimmt werden.
Aus der Fallgeschwindigkeit konnte der Rasius des jeweiligen Teilchens bestimmt werden.
Mit diesem wurden anschließend der Korrekturfaktor der Viskosität und zum Schluss die Ladung des Teilchens berechnet werden.\\
Für den errrechneten Wert der Ladung eines einfach geladenen Teilchens ergibt sich:

\[\boxed{q = \SI{1,56 \pm 0.12}{\cdot 10^{-19}\ \coulomb}}\]

Dieser weist eine Abweichung von $0,35\ \sigma$ zum Literaturwert auf.
Durch die ausreichende Abschätzung der Fehler ist diese Abweichung nicht signifikant.
Der errechnete Wert des Excel-Dokuments ist der Mittelwert aller Einzelladungen von Teilchen mit einer Gesamtladung $\le 6e$.
Dieser ergibt miit der statistischen Abweichung:

\[\boxed{q_X = \SI{1,630 \pm 0.013}{\cdot 10^{-19}\ \coulomb}}\]

Das entspricht einer Abweichung von $2\ \sigma$ zum Literaturwert.
Diese Abweichung ist zwar nicht mehr vernachläsigbar, aber immer noch innerhalb des $3\ \sigma$ Bereichs des Literaturwerts.
Daher kann auch diese Messung als erfolgreich betrachtet werden.

Was hier auffällt ist die Abweichung nach oben, eigentlich sollte die Messung von Excel genauer sein.
Dort wurde allerdings nicht der Fehler der Spannung berücksichtigt, welche in unserer Durchführung stark abgefallen ist. Die Messungen des ersten Wertes
wurden etwa bei der hälfte der Messung gemacht, weshalb dort der Mittelwert der Spannung sehr genau zutrifft.
Ist die Spannung weiter abgefallen Steigt die errechnete Ladung. Deshalb ist der Wert von Excel zu groß.\\ \\


Zusammenfassend war die Messung erfolgreich. Jedoch gäbe es zu verbessern, dass Excel signifikannte Abweichungen der Messgeräte berücksichtigt.
Dadurch können schlechte Messgeräte nicht kompensiert werden und es werden falsche Ergebnisse berechnet.
Ebenfalls wäre ein Netzteil hilfreich, welches die Spannung besser konstant halten kann, damit diese Fehlerquelle reduziert werden kann.