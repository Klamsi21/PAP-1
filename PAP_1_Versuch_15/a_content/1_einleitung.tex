\section{Motivation}



\section{Messverfahren}



\section{Grundlagen aus der Physik}

\subsection{Schiefe Ebene}

Mit $\omega = \frac{a}{r}$ und $l$ als Läge der Bahn und $h$ als Höhe der Bahn

\begin{align}
    ma & = mg\sin(\alpha)- J\omega^2 \\
    \Rightarrow a &=\frac{mg \sin(\alpha)}{m + \frac{J}{r^2}} \\
    \Rightarrow a  &=\frac{mg h}{l (m + \frac{J}{r^2})} \label{eq:a}
\end{align}

\subsection{Trägheitsmomente}

Vollzylinder:

\begin{equation}
    J_V = \frac{1}{2} mr^2
    \label{eq:JVoll}
\end{equation}

Hohlzylinder:

\begin{equation}
    J_H = \frac{1}{2}m (r_2^2 + r_1^2)
    \label{eq:JHohl}
\end{equation}

