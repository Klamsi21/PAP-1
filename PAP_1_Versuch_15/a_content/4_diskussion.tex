\section{Beschleunigungen}
Im Folgenden wird $V$ für den Vollzylinder und $H$ für den Hohlzylinder verwendet.\\

Zuerst wurden die Lichtschranken in einem quadratisch ansteigendem Abstand aufgebaut.
Daraus konnte jeweils eine Messreihe erstellt werden und mithilfe der Steigung der Geraden die Beschleunigung bestimmt werden:

\[\boxed{a_{V1} = \SI{ 0,9910 \pm 0,014 }{\frac{\meter}{\second^2}}}\]

\[\boxed{a_{H1} = \SI{ 0,800 \pm 0,013}{\frac{\meter}{\second^2}}}\]

Durch die Berechnung des Trägheitsmoments aus der gemessenen Masse und des Radius konnte die theoretische Beschleunigung berechnet werden.

\[\boxed{a_{V2} = \SI{1,06 \pm 0,03}{\frac{\meter}{\second^2}}}\]

\[\boxed{a_{H2} =  \SI{0,86 \pm 0,03}{\frac{\meter}{\second^2}}}\]

Daraus ergibt sich für den Vollzylinder die Abweichung $2 \sigma$ und für den Hohlzylinder von $1,8 \sigma$.
Diese Abweichungen sind aufgrund der geringen Fehler akzeptabel. Sie liegen beide innerhalb des $3 \sigma$ Bereichs und können
daher als Aussagekräftig betrachtet werden. Allgemein kann diese Messung also als Erfolgreich gewertet werden.
Kritisch zu betrachten ist jedoch, dass die gemessenen Werte unter den berechneten liegen. Das deutet auf einen systematischen Fehler hin.
Dieser liegt in etwa in der Rollreibung oder dem Luftwiderstand. Bei den Berechnungen wurde die Ebene als ideal betrachtes, was sie ralistisch nicht ist.
Dazu kommt, dass das Anheben der Weggrollsperre nicht instantan geschieht.

\section{Vergleich der Energien}
Anschließend wurden zwei Lichtschranken an die Ebene nach der schiefen Ebene verschoben. Damit konnte die Endgeschwindigkeit bestimmt werden und daraus die kinetische Energie.
Aus der Höhe der Ruhelage der Zylinder konnte dann die potentielle Energie bestimmt werden.
\subsection{Vollzylinder}
Für die Energien des Vollzylinders gilt:
\[\boxed{E_{V kin} = \SI{0,538 \pm 0,015}{\joule}}\]
\[\boxed{E_{V pot} = \SI{0,73 \pm 0,04}{\joule}}\]

Im Idealfall sollten beide Übereinstimmen, da die Energieerhaltung gilt. Dies ist jedoch nicht ganz der Fall.
Die Abweichung beträgt $5 \sigma$.

\subsection{Hohlzylinder}

Für die Energien des Hohlzylinders ergibt sich:

\[\boxed{E_{kin,hohl} = \SI{0,55 \pm 0,13}{\joule}}\]
\[\boxed{E_{pot,hohl} = \SI{0,72 \pm 0,04}{\joule}}\]

Hier ergibt sich ebenfalls eine Abweichung von $4 \sigma$.\\

Die hohen Abweichungen sind wie bei der Beschleunigung durch systematische Faktoren zu erklären.
Einerseits spielen hier Rollreibung und Luftwiderstand eine Rolle, jedoch liegt die größte Fehlerquelle im Übergang zwischen schiefer und ebenener Ebene.
Dort erfuhr der Zylinder einen sehr Ruckartigen schlag, wodurch Energie in den Tisch übergegangen ist. Dadurch wurde die Geschwindigkeit verlangsamt und die Energie verringert.
Selbst wenn man den Übergang Stoßlos macht wirkt bei der Richtungsänderung eine Kraft, auch entgegen der Rollrichtung.\\
Zusammenfassend lässt sich diese Messung mit dem Aufbau nicht verbessern. Aber zu bemerken ist, dass die kinetische Energie geringer ist als die potentielle, was auf den erwähnten Energieverlust hinweist.