Aus dem Versuch konnte einersiets die Periodendauer des ungedämften Pendels und daraus die Frequenz bestimmt werden.
Des weiteren konnte die Frequenz des Motors bei maximaler Amplitude des getriebenen Pendels bestimmt werden.

\[\boxed{f_0 = 0,551 \pm 0,005 \si{Hz}}\]
\[\boxed{f_{res} = 0,553 \pm 0,005 \si{Hz}}\]

Es fällt auf, dass beide Werte nur um $0,028\ \sigma$ voneinander Abweichen.
Daraus lässt sich schließen, dass die Frequenz des Motors nahezu der des ungetriebenen Pendels entsprechen muss, damit die Amplitude maximal wird.
Dieses Phänemen wird als Resonanz beschrieben. Aus dem Ergebnis lässt sich ableiten, dass die Messungen der Frequenzen nicht signifikant voneinander abweichen, die Messung also erfolgreich war.
Bei leight gedämpften Systemen weichen dieser Werte nur leicht voneinader ab, was heir derr Fall ist.
\\ \\


Mithilfe drei verschiedener Methoden konnte jeweils die Dämpfungskonstante des Pendels bestimmt werden.
Zuerst wude $\delta$ über die Halbwertszeit der Amplitude des Pendels bestimmt,
Anschließend über die Halbwertsbreite und zulets über die Resonanzüberhöhung.

\begin{table}[h!]
    \centering
    \begin{tabular}{c c c c c c c c}
        \hline
        $\delta$ & $a$ & $b$ & $c$ & $\sigma_{ab}$ & $\sigma_{bc}$ & $\sigma_{ac}$ \\
        \hline
        1 & $0,122 \pm 0,004$ & $0,067 \pm 0,028$ & $0,10 \pm 0,03$ & $1,9$ & $0,8$ & $0,7$ \\
        2 & $0,295 \pm 0,011$ & $0,255 \pm 0,028$ & $0,22 \pm 0,05$ & $1,8$ & $0,6$ & $1,5$ \\
        \hline

        
    \end{tabular}
    \caption{Verglecih $\delta$'s}
\end{table}

Aufällig ist hier, dass die Werte von $b$ und $c$ gut miteinander übereinstimmen, also nicht signifikante Abweichungen aufweisen, der Wert von Methode $a$ jedoch stärker abweicht.
Methode $a$ beruht auf dem manuellen Ablesen der Amplitude welches eine große Fehlerquelle darstellen kann.
Hier lässt sich vermuten, dass der Ablesefehler nicht ausreichend groß abgeschätzt wurde.
Für eine erneute Durchführung wäre eine Kamera mit ausreichender Bildrate notwendig, um die Amplituden
von schärferen Bildern ablesen zu können. \\
Aufgrund der Abweichungen lässt sich sagen, dass trotzdem alle Messungen erfolgreich waren, wobei Methoden $b$ und $c$ genauer waren als $a$.
Methoden $b$ und $c$ beruhen ebenso auf manuellem Ablesen, jedoch gibt es dort mehrere Durchläufe mit gleicher Amplitude, weshalb ein genaueres Einschätzen der Amplitude und des Fehlers möglich ist.
Da alle Abweichungen innerhalb des $3\ \sigma$ Bereichs liegen, kann die Messung als ausreichen genau und erfolgreich betrachtet werden.




