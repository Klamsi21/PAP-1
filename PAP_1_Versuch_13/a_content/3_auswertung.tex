\section{Aufgabe I}

\begin{table}[h!]
    \centering
    \begin{tabular}{c r r}
        \toprule
        Messung & Gemessene Zeit [s] & Umlaufzeit $T_0$ [s] \\
        \midrule
        1 & $36,32 \pm 0,3$ & $ 1,816 \pm 0,015$ \\
        2 & $36,34 \pm 0,3$ & $ 1,817 \pm 0,015$ \\
        3 & $36,31 \pm 0,3$ & $ 1,816 \pm 0,015$ \\
        \bottomrule
        \multicolumn{3}{c}{$\overline{T_0} = 1,816 \pm 0,015$}
    \end{tabular}
\end{table}
Fehler:

\begin{equation}
    \Delta \overline{T_0} = \sqrt{\sigma^2 + (\Delta T_0)^2}
\end{equation}

Wobei $\Delta T_0$ die Abweichung einer Umlaufzeit ist, also die Reaktionszeit heruntergerechnet auf einen Umlauf und $\sigma$ die
mittlere Abweichung des Mittelwerts ist.