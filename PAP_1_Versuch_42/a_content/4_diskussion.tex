\section{Zusammenfassung}



\section{Wasserwert}

Der berechnete Wasserwert liegt bei $\boxed{W= ( 95 \pm  17)\ \tfrac{\text{J}}{\text{K}}}$.
Verglichen mit dem angegebenen Wert von $70\ \tfrac{\text{J}}{\text{K}}$ ergibt sicht eine Abweichung
von $1,5 \sigma$. Der gemessene Wert liegt noch gut im $3\sigma$ Bereich. Deshalb ist der Wert mit
dem angegebenen vereinbar und kann als genau anerkannt werden.
Die größte Fehlerquelle liegt höchstwarscheinlich in der genauen bestimmung der Wassertemperatur vo der Messung.
Durch den Einschüttvorgang konnte das Wasser Wärme an die Umgebungsluft abgeben wodurch es kälter als erwartet im Kalorimeter angelangt ist.
Daraus resultierte eine kühlere Gleichgewichtstemperatur und dementsprechend ein höherer Wasserwert. Allerdings schwangt die Wärmeisolierung von Gerät zu Gerät weshalb der Vergleich nicht für alle Geräte Sinnvoll ist.

\section{Spezifische Wärme Kalorimeter}
Die Ergebnisse von Aluminium mit $\boxed{c_m = 0,84 \pm 0,03\ \tfrac{\text{J}}{\text{gK}}}$ und Blei $\boxed{c_m = 0,13\pm0,01\ \tfrac{\text{J}}{\text{gK}}}$
sind mit Abweichungen von $2\sigma$ und $0,1\sigma$ als Erfolgreich zu bewerten. Bei der Blei Messung
hätten die Fahlerabschätzungen verringert werden können um das Ergebnis genauer eingrenzen zu können.\\
Auffällig ist hingegen das Ergebnis von Graphit  $\boxed{c_m = 0,77\pm0,02\ \tfrac{\text{J}}{\text{gK}}}$. Dieses erzielt eine Abweichung von $3\sigma$.
Dieser Bereich gilt zwar als akzeptabel jedoch muss das Ergebnis kritisch betrachtet werden.
Einen Fehler könnte der Wasserwert sein, jedoch müsste dieser Fehler auch auf die anderen Werte zutreffen.
Es muss auf einen systematischen Fehler geschlossen werden, bei welchem kleine Fehler in der Versuchdurchführung passiert sind.\\
Besonders auffällig ist ebenfalls die hohe Abweichung von $70\sigma$ bei Graphit von der Dulong-Petit Methode.
Das Problem bei der Dulong Petit methode ist, dass sie bei leichten Elementen wie Kohlenstoff versagt. Das kommt, da die
Schwingungen sehr Hochfrequent sind und deshalb die Energiezustände nicht vollständig angeregt sondern quantisiert sind.

\section{nicht konstante werte}

\section{Spezifische Wärme Stickstoff}
