\section{Motivation}

Bei der Energiegewinnung durch Sorazellen stellt bislang der Energiespeicher eine Begrenzung der Effizienz dar.
Da Solarzellen nur Tagsüber und bei gutem Wetter effizient funktionieren, muss die Energie gut gespeichert werden um auch Nachts oder bei Regen Energie liefern zu können.
Ein Konzept für einen Energiespeicher ist das Speichern der erzeugten Wärme in Salzschmelzen.
Diese sind besonders Effizient aufgrund ihrer hohen Wärmekapazität. Das bedeutet, dass sie viel Wärme aufnehmen können, bevor sich ihre Temperatur drastisch erhöht.
Doch gibt es noch effizientere Wärmespeicher? Dieses Experiment vermittelt die Größe der Wärmekapazität um ein anschauliches Verständnis dieser zu erlangen.

\section{Ziel des Versuchs}

Das Ziel des Versuchs ist es die Größen der Wärmekapazität kennen un einornen zu lernen.
Durch das bestimmen der Wärmekapazität verschiedener Materialien bei verschiedenen Temperaturen
wird vermittelt, dass Wärmekapazitäten Stoff und Temperaturabhängig sind.

\section{Messverfahren}

Verwendet wird in dem ersten Teil der Versuchs ein Wasserkalorimeter.
Um am Ende die Wärmekapazitäten der Stoffe zu bestimmen werden die Gewichte dieser benötigt und zunächst der Wasserwert, also die Wärmekapazität des Kaloriemeters, bestimt.
Dafür wird das Kaloriemeter mit etwa $50 ^\circ C$ warmen Wasser befüllt und in regelmäßigen Zeitabständen die Temperatur bestimmt.\\

Im zweiten Teil werden die großen Massen auf die Siedetemperatur von Wasser erhitzt. Anschließend werden sie in das Kaloriemeter gelassen, welches mit Wasser auf Zimmertemperatur gefüllt wirde.
Darauf wird die sich einstellende Maximaltemperatur abgelesen und daraus mithilfe des Wasserwerts die Wärmekapazität der eintzelnen Proben bestimmt. \\

Zuletzt wird ein ähnliches Prinzip in Stickstoff angewand. Die kleinen Proben befinden sich auf Zimmertemperatur.
Diese werden in ein abgewogenes Dewargefäß mit flüssigem Stickstoff gelassen. Hört der Stickstoff auf zu sieden ist die Gleichgewichtstemperatur erreicht.
Abschließend wird die Massendifferenz des Dewargefäßes bestimmt, welche der Menge des verdunsteten Stickstoffs entspricht.
Damit werden erneut die Wärmekapazitäten berechnet.

hh


\begin{equation}
    W=\frac{c_{w} \left(T - T_{1}\right) \left(  m_{w}- m_{k}\right)}{ T_{2}- T}
    \label{eq:Wasserwert}
\end{equation}

\begin{equation}
    T = 100^\circ \text{C} + 0,0276 \frac{^\circ \text{C}}{\text{hPa}}(p -1014 \text{hPa})
    \label{eq:Siedetemperatur}
\end{equation}

\begin{equation}
    c_x = \frac{Q_V m_V}{m_x(T_1-T_2)}
    \label{eq:cN}
\end{equation}
