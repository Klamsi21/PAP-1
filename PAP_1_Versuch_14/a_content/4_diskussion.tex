\section{Methode 1}
Als Referenz wird hier der Heidelberger Ortswert von $g_H = 9,80984$. Der Fehler wird nicht betrachtet, da die Größenordnungen mit unseren Messwerten
in keinerlei Vergleichbarkeit stehen. \\
Mithilfe des Messens der Periodendauer einer Schwingung konnte die Gravitationsbeschleunigung $g$ bestimmt werden:
\[\boxed{g =\SI{9,83 \pm 0,13 }{\frac{\meter}{\second^2}}}\]

Dafür wurde das Pendel 5 mal 20 mal geschwungen und die Gesamtzeit gemessen aus diesen konnte dann die gemittelte Zeit
für eine Periode bestimmt werden.\\
Das Problem bei dieser Messung ist, dass die Ungenauigkeit durch die Reaktionszeit stark ins Gewicht fällt.
Das vefälscht einerseits den Messwert, jedoch erhöht sich dadurch auch der Fehler, wodurch eine Abweichung von $0,18\ \sigma$
festgestellt werden konnte. Die Messugng ist kritisch zu betrachten, da die reine Abweichung keine genau Aussage zulässt.
Die Abweichung ist hier nicht gerade klein, weshalb die Messung zwar als erfolgreich jedoch nich als ausreichend exakt angesehen wird.

\subsection{Methode 2}
Bei der zweiten Methode wurde zuerst die benötigte Anzahl an Schwingungen berechnet um die Reaktionszeit vernachlässigbar zu machen.
Danach wurde etwa 600 Schwingungen durchgeführt und die Amplitude nach jeweils 20 Schwingungen bestimmt. Darraus ergibt sich die
Dämpfung $\delta$. Es wurden ebenfalls Trägheitsmoment der Kugel und des Fadens berücksichtigt, sowie die nicht-linearität der Auslenkung.\\
Nach Berücksichtigung dieser Korrekturen ergibt sich für $g$:
\[\boxed{g = \SI{9,835 \pm 0,013}{\frac{\meter}{\second^2}}}\]

Es ist eine Abweichung zum Literaturwert von $2,2\ \sigma$ festzustellen. Unter Berücksichtigung aller eingerechneten Faktoren
hätte die Abweichung kleiner sein müssen. Da jedoch zwischen beiden gemessenen Werten eine Abweichung von $0,015\ \sigma$ vorliegt, ist auf einen systematischen Fehler zu schließen.
Zu vermuten ist, dass durch die Vielzahl an Messungen einerseits der Dämpfungskoeffizient ungenau bestimmt wurde, da die
Messwerte gegen Ende der Reihe nicht mehr als linear auf der logarithmischen Skale betrachtet werden könenn. Andererseits
sorgt die gleiche Fehlerquelle für eine Fehlmessung der Gesamtzeit. Da die Amplituden stärker als erwartet abgenommen haben,
lässt sich erschließen, dass auch die Periodendauern gegen Ende der Messung unerwartet kurz geworden sind.
Da zu bestimmung von $g$ jeweils durch $T_0^2$ geteilt wird sind beide Werte gleichermaßen von der Fehlmessung betroffen
was die gleichmäßige Abweichung erklären würde. \\ \\
Abschließend lässt sich sagen, dass die Messung einerseits gut gelungen ist, da beide Werte gut miteinander übereinstimmenm, es jedoch einen gravierenden Fehler in der Zeitmessung
gab, weshalb weide werte systematisch vom Literaturwert abweichen.
