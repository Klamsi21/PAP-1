\section{Diagramm 1}
Im ersten Diagramm wurde der Druck der Fixpunkte des Gasthermometers zu deren Temperaturen aufgetragen.
Daraus wurde eine Trendgerade erstellt mit welcher die Temperatur von flüssigem Stickstoff und die das absoluten Nullpunkts ermittelt werden konnte.
Bei der Auswertung von Diagramm 1 fällt auf, dass die gemessene Temperatur des Stickstoffs
\[T_{N_2} = -202 \pm 7 ^\circ C\]
um $1 \sigma $ von dem Literaturwert abweicht. Dies liegt noch im $3\sigma$ Bereich und die MEssung kann deshalb als genau angesehen werden.
Ebenfalls liegt der Wert des absoluten Nullpunkts mit 
\[T_0= -275 \pm 5 ^\circ C\]
im $0,5 \sigma $ Bereich und kann ebenfalls als genau angenommen werden.
Beide Fehler sind ausreichend klein um die Ergebnisse als aussagekräftig zu interpretieren.

\section{Diagramm 2}

Diagramm zwei wurde analog zu Diagramm 1 erstellt, bis auf den Unterscheid, dass hier der Literaturwert der Temperatur des Stickstoffs als weiterer Fixpunkt verwendet wurde.
Es fällt auf, dass die abgelesenen Temperaturwerte abseits der Fixpunkte systematisch nach oben abweichen um  $2,5-5 ^\circ C$.
Es ist auf einen systematischen Fehler zu schließen, welcher sich warscheinlich auf das Einstellen der Temperatur beläuft. Es ist
zu vermuten, dass sich das Flüssigkeitshermometer nicht zu Ende gesetzt hat bevor die Messung vollzogen wurde. Aus den adneren Diagrammen wäre ebenfalls zu vermuten,
dass das Flüssigkeitshermometer eine systematische Ungenauigkeit eventuell durch Eichung hervorruft.\\

Die gemessene Siedetempereatur von Kohlenstoffdioxid betrug:
\[T_{CO_2}= - 66 \pm 1 ^\circ C\]
Da kein Literaturwert vohanden war lässt sich auch keine Abweichung bestimmen.

\section{Diagramm 3}
Im dritten Diagramm wurde der gemessene Widerstand des Platinthermometers auf die gemessene Temperatur des Flüssigkeitthermometers aufgetragen.
Daraus wurde eineut eine Trendgerade erstellt und die Steigung bestimmt umden Zusammenhang zwischen Widerstand und Temperatur zu bestimmen.

Bei der Auswertung von Diagramm 3 fällt auf, dass der Widerstand bei $100 ^\circ C$ mit $131 \pm 1 \Omega$
sehr stark von dem Trend der restlichen Werte abweicht. Zu vermuten ist ein systematischer Ablesefehler
bei welche der Wert des Multimeters falsch notiert wurde. Deshalb wurde dieser Wert für die Geradeneinzeichnung und Fehlerrechnung nicht berücksichtigt.\\
Der berechnete Koeffizient $m = 0,45 \pm 0,14  \tfrac{\Omega}{K}$ liegt im $0,4 \sigma$ Bereich des Literaturwerts, wobei hier kritisch betrachtet werden muss,
dass der Fehler $\approx 30 \% $ des berechneten Werts beträgt.
Daraus lässt sich schließen, dass die Messung ausrechend genau ist, aber deutlich genauer durchgeführt werden kann.


\section{Diagramm 4}

In Diagramm 4 wurde die Temperaturen aus Diagramm zwei gegen die Temperaturen des Pyrometers aufgetragen. Im Idealfall messen diese die
gleichen Temperaturen, weshalb das Verhältnis, also die Steigung $= 1$ sein sollte.
Der abgelesene Wert betrug:
\[ m = 0,94 \pm 0,14\]
Das entspricht einer Abweichung von $0,4 \sigma$ zum angegeben Literaturwert.
Bei der Druchführung flie auf, dass die Messungen des Pyrometers nicht auf die des Flüssigkeitthermometers passten.
Anzumerken ist hier, dass die Messwerte trotzdem ausreichend gut übereinstimen und die Messung gut im $3 \sigma$ Bereich liegt.
\\
Es ist darauf zu schließen, dass der Fehler bei dem Flüssigkeitthermometer lag, da es warscheinlicher ist, dass die einzelne Messung falsch ist,
als sowohl die des Gasthermometers als auch die des Pyrometers.

% \section{Verebsserung}

% Besonders bei Diagramm 2 fallen große Fehler auf. Es wäre Sinnvoll gewesen
% sich bei der Bestimmung der Temperaturstufen nicht auf

